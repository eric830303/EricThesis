\begin{abstractzh}
\hfill\break

\begin{center}

\Large{藉由回收老化引起的時鐘偏差使老化變得有用進而改善電路可靠度}
\end{center}
\hfill\break
研究生: 曾天鴻 $\hfill$指導教授:吳凱強 教授





\hfill\break

\begin{center}
國立交通大學資訊科學與工程研究所 \\

\hfill\break
摘\hspace{2cm}要\\
\end{center}


\hfill\break

電子元件老化使得電路性能和壽命顯著損失,這是使電子元件可靠性降低的主要因素。在這篇論文中,我們建議通過操縱和回收這些時變偏斜來利用老化引起的時鐘偏差(意即,使電路能容忍更多老化),藉以補償因老化導致的邏輯電路之性能下降。我們的目標是在電路中分配可實現/合理的老化引起的時鐘偏差,使得可以容忍由於老化導致的性能下降,(意即,可以最大化電路壽命)。藉由實驗,平均可以達到24.95%老化耐受性。此外,我們在時鐘緩衝器上採用$V_{th}$分配,以進一步容忍老化引起的邏輯電路效能降級。當$V_{th}$分配應用於上述老化操作之上時,平均老化耐受性可以提高到37.61%。

\end{abstractzh}



\begin{abstracten}
%\hfill\break

\begin{center}
\Large{Making Aging Useful by Recycling Aging-Induced Clock Skew}
\end{center}
\hfill\break

Student: Tien-Hung Tseng $\hfill$ Advisor: Prof. Kai-Chiang Wu

\bigskip

\begin{center}
Institute of Computer Science and Engineering\\
National Chiao-Tung University\\
\bigskip

ABSTRACT
\end{center}

\bigskip

\begin{abstract}
Device aging, which causes significant loss on circuit performance and lifetime, has been a primary factor in reliability degradation of nanoscale designs. In this paper, we propose to take advantage of aging-induced clock skews (i.e., make them useful for aging tolerance) by manipulating these time-varying skews to compensate for the performance degradation of logic networks. The goal is to assign achievable/reasonable aging-induced clock skews in a circuit, such that its overall performance degradation due to aging can be minimized, that is, the lifespan can be maximized.  On average, 24.95\% aging tolerance can be achieved with insignificant design overhead. Moreover, we also apply V\textsubscript{th} assignment to further mitigate the aging-induced degradation of logic networks. Averagely, 37.89\% aging tolerance can be achieved when aging manipulation and V\textsubscript{th} assignment are applied.
\end{abstract}


\end{abstracten}
