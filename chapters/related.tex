\section{RELATED WORK AND PAPER CONTRIBUTION}
\label{sec:related}
\subsection{Previous Work on Aging-Aware Optimization}
\label{subsec:rw}
To deal with aging phenomena, traditional design methods adopt guard-banding by adding extra timing margins, which in practice imply over-design and may be expensive. To avoid overly conservatism, the mitigation of aging-induced performance degradation can be formulated as a timing-constrained area minimization problem with consideration of aging effects. Existing aging-aware techniques basically follow this formulation. A novel technology mapper considering signal probabilities for NBTI was developed in~\cite{kumar2007nbti}. On average, 10\% area recovery and 12\% power saving are accomplished, as compared to the most pessimistic case assuming static NBTI on all PMOS transistors in the design. The authors of~\cite{paul2006temporal} proposed a gate sizing algorithm based on Lagrangian relaxation. An average of 8.7\% area penalty is required to ensure reliable operation for 10 years. Other methods related to gate or transistor sizing can be found in~\cite{wang2007efficient, yang2007combating}.

Aforementioned research work focus on mitigating the aging of logic networks only. There are some studies~\cite{huang2013low, chakraborty2013skew,chen2013novel} addressing the aging problem of clock networks. Methodology of~\cite{chen2013novel} is based on $V_{th}$ assignment for clock buffers. The authors of ~\cite{huang2013low} and ~\cite{chakraborty2013skew} explore the use of alternative clock gating cells for clock-gated designs. On the other hand, two new clock gating cells were presented in~\cite{lai2014bti} to balance the delay degradation of clock signal propagation, which reduces aging-induced clock skews between ungated and gated clock branches. However, the studies~\cite{huang2013low, chakraborty2013skew,chen2013novel,lai2014bti} aim to minimize aging-induced clock skews instead of making it useful.
%Aforementioned methods focus on mitigating the aging of logic networks only. To address the aging problem of clock networks,~\cite{chen2013novel, huang2013low, chakraborty2013skew} aim to minimize aging-induced clock skews by balancing the aging effects on various clock sub-networks. While~\cite{chen2013novel} is based on $V_{th}$ assignment for clock buffers,~\cite{huang2013low},~\cite{chakraborty2013skew} explore the use of alternative clock gating cells for clock-gated designs. On the other hand, two new clock gating cells were presented in~\cite{lai2014bti} to balance the delay degradation of clock signal propagation, which reduces aging-induced clock skews between ungated and gated clock branches.

\subsection{Paper Contribution}
\label{subsec:pc}
In this paper, we propose an optimization framework for aging tolerance. Our proposed framework manipulates the rates of aging on different clock branches, by changing the duty cycle of a clock waveform delivered to each of the clock branches. In addition, we employ $V_{th}$ assignment on clock buffers to achieve additive aging tolerance. The contributions and advantages of this work are threefold:
%In this paper, we propose an optimization framework, MAUI (Making Aging Useful, Intentionally), for aging tolerance. Our proposed framework manipulates the rates of aging on different clock branches, by changing the duty cycle of a clock waveform delivered to each of the clock branches. The contributions and advantages of this work are threefold:




\begin{itemize}
\item \textbf{\textit{Exploration of aging-induced clock skews for aging tolerance:}} Existing work on addressing aging-induced clock skews mainly attempts to minimize the skews. This paper presents the first work on \textit{exploring/recycling \enquote{useful} clock skews (i.e., making them useful)} for aging tolerance\footnote[9]{We do not minimize \enquote{absolute} performance degradation by directly mitigating the aging of logic networks; instead; we minimize \enquote{effective} performance degradation by manipulating the aging of clock networks to tolerate expected logic's aging based on timing borrowing, as a result of aging-induced clock skews. Of course, one can mitigate the logic's aging itself by using existing techniques~\cite{kumar2007nbti, paul2006temporal, wang2007efficient, yang2007combating} before applying the proposed framework. This is however beyond the scope of this work and thus not particularly addressed in this paper.}.
\item  \textbf{\textit{Problem formulation based on Boolean satisfiability and optimal solutions:}} The proposed formulation of making aging useful is transformed into a Boolean satisfiability (SAT) problem, and its optimal solution can be efficiently found by a SAT solver such as MiniSat.
\item \textbf{\textit{Low design overhead and little design modification:}} Restrained by the synthesized clock tree whose topology and structure are basically determined, our post-CTS (clock tree synthesis) framework does not involve aggressive modification and thus does not incur significant design overhead.
\end{itemize}



%\begin{itemize}
%\item \textbf{\textit{Exploration of aging-induced clock skews for aging tolerance:}} Existing work on addressing aging-induced clock skews mainly attempts to minimize the skews. This paper presents the first work on \textit{exploring \enquote{useful} aging-induced clock skews (i.e., making them useful)} for aging tolerance\footnote[1]{We do not minimize \enquote{absolute} performance degradation by mitigating the aging of logic networks; instead; we minimize \enquote{effective} performance degradation by manipulating the aging of clock networks to \enquote{tolerate} expected logic's aging based on time borrowing, as a result of aging-induced clock skews. Of course, one can mitigate the logic's aging itself by using existing techniques~\cite{kumar2007nbti, paul2006temporal, kang2007efficient, yang2007combating} before applying the proposed framework. This is however beyond the scope of this work and thus not particularly addressed in this paper.}.
%\item  \textbf{\textit{Problem formulation based on Boolean satisfiability and optimal solutions:}} The proposed formulation of making aging useful is transformed into a Boolean satisfiability (SAT) problem, and its optimal solution can be efficiently found by a SAT solver such as MiniSat.
%\item \textbf{\textit{Low design overhead and little design modification:}} Restrained by the synthesized clock tree whose topology and structure are basically determined, our post-CTS (clock tree synthesis) framework does not involve aggressive modification and thus does not incur significant design overhead.
%\end{itemize}